\chapter{群,环,域}

在随后的四个章节里,我们将回顾基本的代数结构(群,环,域,向量空间),向量空间是
我们介绍的重点。一些基本的线性代数符号,诸如向量空间,子空间,线性组合,线性无关,基,
商空间,线性映射,矩阵,基变换,内积,线性转化,对偶空间,超空间,线性映射的转化
都将给予介绍。
\section{群,子群,紧集}
实数$R$有两个操作,$+ : R\times R\to R$(加法)和$*:R\times R\to R$(乘法),满足以下性质:使得实数
集$R$在加法操作下是一个阿贝尔群,去除0元素的$R-\{0\}=R^*$在乘法操作下是一个阿贝
尔群。我们回顾群的定义:
\begin{definition}
  一个集合$G$是一个群,如果它满足二元操作:对任意两个元素$a,b\in G$都有相应的一个
  元素$a\cdot b\in G$,其中操作是可结合的,且G有一个基本元$e$,$G$中每一个元素都可逆。显然,这意味着以下式子对任意
  $a,b,c\in G$ 成立:

  (G1) $a\cdot(b\cdot c) = (a\cdot b)\cdot c $ \hfill(结合律)

  (G2) $a\cdot e=e\cdot a=a$ \hfill (基本元)

  (G3) 对于任意$a\in G$,存在逆元$a^{-1}\in G$,因此$a\cdot a^{-1}=a^{-1}\cdot a=e$ \hfill (可逆
  性)
\end{definition}
一个群如果是阿贝尔(或者说可交换)的:
\[
  a\cdot b = b\cdot a \quad \textrm{for all } a,b\in G.
  \]
一个集合如果有二元运算$M\times M\to M$并且有一个基本元e满足条件(G1)和(G2),那么我们
称之为幺半群。例如,自然数$N=\{0,1,\ldots,n,\ldots\}$,在加法操作下是一个幺半群。但是它并
不是一个群。

一些群的例子如下:
\begin{example}
  
  1.整数集$Z = \{\ldots,-n,\ldots,-1,0,1,\ldots,n,\ldots\}$在加法操作下是一个阿贝尔群,有基本元0。
  但不包含0的整数集$Z^*=Z-{0}$在乘法操作下并不是一个群。

  2.有理数$Q(p/q,其中p,q \in Z, q\neq 0)$在加法操作下是一个阿贝尔群,有基本元0。不
  含0元素的$Q^* = Q-{0}$在乘法操作下也是一个阿贝尔群,有基本元1.

  3.给定任意非空集合 $S$,集合$S$上的双射操作集合$f:S\to S$(也称之为集合$S$的置换),是在复合
  函数操作下的一个群(例如,函数$f,g$的乘法复合是$g\cdot f$),有一个基本元,即元素自
  身到自身映射。集合$S$如果有超过两个元素,那么S的置换操作集合不是一个阿贝尔群。
  集合$S=\{1,\ldots ,n\}$上的置换群称为n个元素的对称群,表示为$\delta_n$。

  4.对于任意正整数$p \in N$,定义一个在整数集$Z$上的关系,表示为$m \equiv n \pmod{p}$,如
  下
  \[
    m \equiv n\pmod{p} \quad\textrm{iff}\quad m-n=kp\quad \textrm{for some } k \in Z
  \]
  读者很容易验证这是一个等价关系,它兼容于加法和乘法操作,这意味着如果$m_1 \equiv n_1
  \pmod{p}$且$m_2 \equiv n_2$,那么$m_1+m_2=n_1+n_2 \pmod{p}, m_1m_2 \equiv n_1n_2 \pmod{p}$。
  所以,我们可以定义等价类$\pmod(p)$的集合上的加法和乘法操作:$[m] + [n] = [m+n]\textrm{,}
  [m]\cdot[n]= [mn]$

  读者很容易验证剩余类$\pmod(p)$上的加法操作揭示一个阿贝尔群结构,其中有[0]作为
  0元素。这个群表示为$Z/pZ$

  5.$n\times n$的可逆实系数矩阵(或者复系数矩阵)集合,在矩阵乘法操作下是一个群,有单
  位元矩阵$I_n$。也称之为一般线性群,表示为$GL(n,R)(GL(n,C))$
  
  6.$n\times n$且行列式$det(A)=1$的可逆实系数矩阵(或者复系数矩阵)的集合,在矩阵乘法操作下是一个群,有单
  位元矩阵$I_n$。也称之为特殊线性群,表示为$SL(n,R)(SL(n,C))$

  7.满足$QQ^T=Q^TQ=I_N,Q^{-1}=Q^T$的$n\times n$的实系数矩阵集合,在矩阵乘法操作下是一个群,有单
  位元$I_n$。称之为正交群,表示为$O(n)$

  8.满足$QQ^T=Q^TQ=I_N,det(Q)=1,Q^{-1}=Q^T$的$n\times n$的实系数矩阵集合,在矩阵乘法操作下是一个群,有单
  位元$I_n$。称之为特殊正交群,表示为$SO(n)$

\end{example}

例子(5)-(8)中涉及的群除了$SO(2)$是一个阿贝尔群之外其他当$n\ge 2$时不是阿贝尔
群($O(2)$不是阿贝尔的).

我们习惯于将阿贝尔群$G$加法操作下,$a\in G$的逆元$a^{-1}$表示为$a^{-1}$

群的基本元是唯一的。事实上,我们可以证明一个更一般的情况如下:
\begin{proposition} %\label{thm:unique_element}
  如果一个二元运算:$M\times M\to M $是可结合的并且如果$a' \in M $是一个左基本元$e''
  \in M $是一个右基本元,这意味着:
\[
    e'\cdot a= a \quad \textrm{for all } a\in M \qquad (G2l)
\]
\[
    a\cdot e''= a \quad \textrm{for all } a\in M \qquad(G2r)
\]
\[
    那么 e'=e''
\]
\end{proposition}
证明:令$a = e''$由(G2l)得到
\[
  e'\cdot e''=e''
\]
令$a = e'$由(G2r)得到
\[
  e'\cdot e''=e'
\]
因此得证
\[
  e'=e'\cdot e''=e'
\]
%\ref{thm:unique_element}
定理表明幺半群的基本元是唯一的。因为每一个群同时也是幺半
群,所以群的单位元也是唯一的。更进一步,群中的每一个元素都有一个唯一的逆元。这个
结果可以从以下更一般的情况得出:
\begin{proposition}
  一个幺半群有基本元$e$,如果一些元素$a\in M$有左逆元$a' \in M$并且有右逆元$a'' \in
  M$,这表示为
\[
  a'\cdot a = e \qquad (G3l)
  \]
\[
  a\cdot a'' = e \qquad (G3r)
  \]
  那么$a'=a''$

  \emph{证明}:基于(G3l)和$e$是基本元,我们有
\[
(a'\cdot a)\cdot a'' = e\cdot a'' = a''
\]
类似地,基于(G3r)和$e$是一个基本元,我们有
\[
a'\cdot(a \cdot a'') = \cdot a'\cdot e = a'
\]
然而,$M$是一个幺半群,操作是可结合的,所以我们有
\[
a'=a'(a\cdot a'')= (a'\cdot a) \cdot a'' = a''
\]
得证

注意:公理(G2)和公理(G3)可以弱化为(G2r)和(G3r)(右基本元的存在性和元素的
右逆元的存在性)。可以尝试证明一下从公理(G2r)和公理(G3r)推出(G2)和(G3)
\end{proposition}

\begin{definition}
  如果群$G$有n个元素,我们称之为$n$阶群。如果群$G$是无限的,那么我们说群$G$有无
  穷阶。群$G$的阶通常表示为$|G|$(如果群是有限的)
\end{definition}
给定一个群$G$,对其中任意两个子集 $R,S\in G$,我们令
\[
  RS = \{r\cdot s| r\in R, s \in S\}
\]
更一般地,对于任意$g\in G, R=\{g\}$,我们表示为
\[
  gS = \{g\cdot s| s \in S\}
\]
相似地,如果$S=\{g\}$,我们表示为
\[
  Rg = \{r\cdot g| g\in R\}
\]
从现在开始,我们省略乘法符号,用$g_1g_2$表示$g_1\cdot g_2$。
\begin{definition}
给定$G$群,对于任意$g\in G$,定义$L_g$为$g$的左平移:任意$a\in G, L_g(a) = ga$,同
理定义$g$的右平移$R_g$:$R_g(a) = ag$,任意$a \in G$
\end{definition}
以下简单的定理经常被使用:
\begin{proposition}
  给定一个群G,其平移变换$L_g, R_g$是双射
\end{proposition}
\emph{证明}:我们只证明$L_g$,$R_g$的证明是类似的。
如果$L_g(a)=L_g(b)$,那么$ga=gb$,我们同时左乘以$g^{-1}$,我们得到$a=b$,所以
$L_g$是一个单射函数。对任意$b \in G$,我们有$L_g(g^{-1})=gg^{-1}b=b$,所以$L_g$是
满射。因此$L_g$是一个双射。
%%buizz 满射的证明不太明白。
\begin{definition} %\label{thm:subgroup_definition}
  给定一个群$G$,其子集$H$称之为$G$的子群,当且仅当:
   
(1) 群$G$的基本元也包含在$H$里($e\in H$)

(2) 任意$h_1, h_2 \in H$,我们有$h_1h_2 \in H$

(3) 任意$h\in H$,我们有$h^{-1} \in H$
\end{definition}
以下定理的证明留作练习
\begin{proposition}
给定一个群$G$,子集$H \subseteq G$是$G$的子群,当且仅当$H$非空,并且任意$h_1,h_2 \in
H,h_1h_2^{-1} \in H$
\end{proposition}
如果群$G$是有限的,那么下列的命题成立
\begin{proposition}
  给定一个有限群$G$,子集$H$称为$G$的子群,当且仅当

  (1) $e\in H $

  (2) $H$ 在乘法操作性是闭合的。

\end{proposition}
%\ref{thm:subgroup_definition}
\emph{证明}: 我们只需要证明定义的条件3成立。对任意
$a\in H$,因为其左平移$L_a$是双射,所以到$H$上的映射是单射。同时,$H$是一个有限集,
所以映射是一个双射。因为$e\in H$,那么有唯一的$b\in H$满足$L_a(b)=ab=e$。如果
$a^{-1}$是$a\in G$中的逆,那么我们有$L_a(a^{-1}) = aa^{-1}=e$,根据$L_a$的单射性,
我们有$a^{-1}=b\in H$。

\begin{example}

  1.任意整数$n\in Z$,集合
  \[
    nZ=\{nk | k \in Z\}
  \]
  是群$Z$的一个子群。

  2.矩阵集合$GL^+(n,R)=\{A\in GL(n,R)|det(A)>0\}$是群$GL(n,R)$的一个子群。

  3.群$SL(n, R)$是群$GL(n,R)$的子群

  4.群$O(n)$是群$GL(n,R)$的子群
  
  5.群$SO(n)$是群$O(n)$的子群,也是$SL(n, R)$的子群

  6.不难证明每一个$2\times 2$的选择矩阵$R\in SO(2)$可以写为
\[
R=
  \left( \begin{array}{cc}
           \cos\theta & -\sin\theta \\
           \sin\theta & \cos\theta \\
         \end{array} \right), \quad 0 \leq \theta<2\pi
\]
其中$SO(2)$可以看作$SO(3)$的子群,即矩阵形式为
\[
R=
  \left( \begin{array}{cc}
           \cos\theta & -\sin\theta \\
           \sin\theta & \cos\theta \\
         \end{array} \right)
\]
视为
\[
Q=
  \left( \begin{array}{ccc}
           \cos\theta & -\sin\theta &0 \\
           \sin\theta & \cos\theta  &0 \\
           0     & 0      &1
         \end{array} \right)
\]

 7.上三角矩阵$\left(\begin{array}{cc}
               a & b\\
               0 & c
\end{array}\right) a,b,c\in R, a,c\neq 0$
是群$GL(2,R)$的一个子群

 8.含有四个矩阵$V=\left(\begin{array}{cc}
               \pm & 0\\
               0   & \pm
\end{array}\right)$的集合是群$GL(2,R)$的一个子群,也称为克莱因四元群。
\end{example}

\begin{definition}
  如果H是G的一个子群,g是G的任意元素,那么形如$gH$的集合称为$H$在$G$上的左陪集,
  形如$Hg$的集合成为$H$在$G$上的右陪集(相似地,右陪集也一样)。H的左陪集可以推出等价关系$\sim$定义如下:对
  任意$g_1,g_2\in G$
  \[
    g_1\sim g_2 \quad textrm{当且仅当} g_1H=g_2H 
    \]
    (类似地,$g_1\sim g_2 \quad textrm{当且仅当} Hg_1=Hg_2)$。显然,$\sim$是一个等价关系。
\end{definition}
    现在,我们介绍以下定理








