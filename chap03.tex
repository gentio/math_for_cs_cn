\chapter{向量空间,基,线性映射}
\section{向量空间}
%\label{sec:label}
当$n\ge 1$时,用$R^n$表示n元组$x=(x_1,\ldots ,x_n)$的集合,$R^n$加法操作可以表示如下:
\[
  (x_1,\ldots,x_n)+(y_1,\ldots,y_n)=(x_1+y_1,\ldots,x_n+y_n)
\]
我们同时定义操作如下:$R\times R^n\to R^n$如下:
\[
  \lambda\cdot(x_1,\ldots,x_n) = (\lambda x_1,\ldots,\lambda x_n)
  \]
  由向量空间引出的代数结构有一些有趣的性质。向量空间的定义如下:
\begin{definition}
  给定一个域$K$(配备了加法和乘法操作)。一个在K上的向量空间(K维向量空间),是向量
  的集合,集合上满足两个操作:向量加法\footnote{加法符号$+$在这里被复用,同时
    表示域$K$上的加法和空间$E$上的向量加法。当+被使用时其功能可以从上下文得出}$(+:E\times E\to E)$和标量乘法$K\times E\to E$,满足

  一下条件:任意$\alpha,\beta\in K,\,u,v\in E$

  (V0) $E$是在加法操作下是一个阿贝尔群,有基本元0 \footnote{符号0在这里也被复用,
    同时表示K上的的零(标量)和E上的基本元(零向量)。可能较少有疑惑,有时候偏向于使
    用\textbf{0}表示零向量}

  (V1) $\alpha \cdot (u+v)=(\alpha\cdot u)+(\alpha\cdot v)$

  (V2) $(\alpha + \beta)\cdot u= (\alpha\cdot u)+(\beta\cdot u)$

  (V3) $(\alpha*\beta)\cdot u=\alpha\cdot(\beta\cdot u)$

  (V4) 1$\cdot u = u$

  在(V3)中,*表示域$K$上的乘法操作。
\end{definition}

给定$\alpha \in  K , v\in E$,元素$\alpha \cdot v$时常表示为$\alpha v$。域$K$也常称为标量域

在接下来的章节里,除非特别指定或者谈及别的不同的域,我们假定所有的K维向量空间是
定义在域K上的空间。因此,当我们谈及向量空间时,我们指代的是K维向量空间。大部分情
况下,K域指的是实数域

在(V0)里,一个向量空间总是包含空向量0,所以这是一个非空集合。从(V1),我们知道$\alpha
\cdot 0 = 0,\;\alpha\cdot(-v)=-(\alpha\cdot v)$。从(V2)我们得到$0\cdot v = 0,\;(-\alpha)\cdot v=-(\alpha \cdot v)$

从公理我们可以得到另外一个重要的结论:对任意$u \in E,\;\lambda \in K$如果$\lambda \neq 0,\;
\lambda\cdot u = 0$,那么$u=0$

事实上,因为$\lambda \neq 0$,那么存在乘法逆元$\lambda^{-1}$,从而根据$\lambda\cdot u = 0$,我们可以
得到
\[
  \lambda^{-1}\cdot (\lambda\cdot u)= \lambda^{-1}\cdot0
  \]
  然而,我们知道$\lambda^{-1}\cdot0=0$,从(V3)和(V4),我们有
  \[
  \lambda^{-1}\cdot (\lambda\cdot u)= (\lambda^{-1}\cdot\lambda)\cdot u = 1\cdot u = u
  \]
  因此我们推出$u=0$

\textbf{注意}:有人质疑公理(V4)是否真的有必要,他能否从其他公理推出?答案是不可以。
例如:我们定义$E=R^n,\;R\times R^n\to R^n$
\[
  \lambda \cdot (x_1,\ldots ,x_n) = (0,\ldots,0)
  \]
  其中$x_1,\ldots ,x_n) \in R^n\quad,\lambda \in R$。公理(V0)-(V3)都满足,但公理4不成立。更一般
  的例子可以通过使用现在还未介绍的基的概念。
  